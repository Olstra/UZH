\documentclass{article}
\usepackage{enumitem}

\title{ \line(1,0){415} \\ Formale Grundlagen der Informatik I -\\ Assignment 1 \theexercise\\
\line(1,0){415}}
\author{Oliver Strassmann, 15-932-726}

\begin{document}
\maketitle



\section{Sets and  Subsets}

\begin{enumerate}[label=\alph*]
	\item)
	\begin{enumerate}[label=\roman*]
		\item The set of all numbers that are real but not rational.
		\item The set of all negative integers that are strictly bigger than -6.\\ 
	\end{enumerate}

	\item)
	\begin{enumerate}[label=\roman*]
		\item No. The empty set contains no elements, if it contained an element, it wouldn't be empty anymore. Therefore, the empty set does not contain a set that is empty ($\emptyset$).
		\item It contains two elements, 1 and 2, since we only count different elements.
		\item 3: The number 0, the set containing the number 0 and the set containing the set containing the number 0. 
		\item No. The elemts are: the set containing the number 2, ... . $\lbrace$2$\rbrace$ != 2
		\item Yes. The set containing a zero is an element of the set.
		\item No. The only element of the set is a set containing 1 and 2, which is not the same as the integer 2.
	\end{enumerate}	
\end{enumerate}


\section{Relations and Functions}

\begin{enumerate}[label=\alph*]
	\item) A function has exactly one output for every input of the domain. In a relation, a single input item can be related to multiple outputs or to none. (Exception undefined behaviours for functions, e.g. dividing by 0 ). 
	\item) U = $\lbrace$ (2,5), (4,3), (4,5) $\rbrace$ V = $\lbrace$ (2,1), (4,1), (4,3) $\rbrace$ W = $\lbrace$  $\rbrace$
		i.) for the third
		ii.) U and V are not functions, multiple outputs for input = 4. 
	\item) Figure 1: Not a function, multiple arrows from 4. Figure 2: It is a function, every item in X has exactly one arrow pointing to an element of Y. Figure 3: Not a function, 2 has no arrow.


\section{Logical Equivalence}


\begin{enumerate}[label=\alph*]
	\item)
	\begin{enumerate}[label=\roman*]
	\item It's a contradiction:\\
	
	( ( \textbf{p $\land$ $\neg$q $\land$ } $\neg$r ) v ( \textbf{p $\land$ $\neg$q $\land$ } r ) ) $\Leftrightarrow$ ... \\
	p $\land$ $\neg$q $\land$ ( \textbf{$\neg$r v r }) $\Leftrightarrow$ ... \\
	p $\land$ $\neg$q $\Leftrightarrow$ $\neg$( p v q ) \\
	\textbf{p} $\land$ $\neg$q $\Leftrightarrow$ \textbf{$\neg$p} $\land$ $\neg$q \\
	p $\Leftrightarrow$ $\neg$p
	\\\\
	The biconditional statement can never be true since the left and right side are never either both true or both false.\\  
	\item This is a tautology. Because of de Morgans Law we get the following statement:\\
(p v q) v $\neg$p v $\neg$q which is true for all combinations of p and q in a truth table.

	\end{enumerate}


	\item) The statements are not mutualy excluding. The second statement is true  whenever Oliver doesn't speak German. The first statement does not state wheather he does or not, so it does not exclude the second statement.

\end{enumerate}


\section{Conditional Statements}

\begin{enumerate}[label=\alph*]
	\item)
	\begin{enumerate}[label=\roman*]
		p: it walks like a duck \\
		q: it talks like a duck \\
		r: it is a duck 
		\item p $\land$ q $\rightarrow$ r
		\item ($\neg$p v $\neg$q) $\oplus$ r
		\item $\neg$p $\land$ $\neg$q $\rightarrow$ $\neg$r \\
		\\
		The first statement and the third are equivalent, iii. is the contrapotitive of i. . 
	
	\end{enumerate}
	
	\item) This is a conditional statement. Let p be "compound X is boiling" and "its temperature is at least 150°C" is q.
	We know then, that if not p then not q, therefore statement ii. is correct.
	Statement i. is also correct, because it is just another way of formulating the statement in this task.
	Regarding statement iii., it is a necessary condition. Stating it is a sufficient condition would mean that X can also boil if its temperature is less that 150, which is not the case, therefore this statement is false.
	
	\end{enumerate}
\end{enumerate}




\end{document}
