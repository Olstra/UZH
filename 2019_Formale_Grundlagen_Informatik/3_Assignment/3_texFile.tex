\documentclass{article}
\usepackage{enumitem}

\title{ \line(1,0){415} \\ Formale Grundlagen der Informatik I -\\ Assignment 03
\line(1,0){415}}
\author{Oliver Strassmann, 15-932-726}

\begin{document}
\maketitle

\section{Sequences and Sums}
\begin{enumerate}[label=\alph*)]
	\item 
			\begin{enumerate}[label=\roman*.]
				\item 
				\item
				\item
			\end{enumerate}

\end{enumerate}

\section{Binomial Coefficient}
\begin{enumerate}[label=\alph*)]
	\item 
	\item
\end{enumerate}

\section{Mathematical Induction Proofs}
\begin{enumerate}[label=\alph*)]
	\item It is well suited for 
	\item First you make a hypothesis. Then you perform the Base step, followed by the Inductive step. Then make the conclusion. 
	\begin{enumerate}[label=\roman*.]
		\item
		\item
	\end{enumerate}
	\item
	\item
	\begin{enumerate}[label=\roman*.]
		\item p(2)\\2^n < (2+1)! \\
			4 < (3*2*1) \\
			4 < 6 \\
			The statement is true, so P(2) is true.
		\item P(k):\\ 2^k < (k+2)!
		\item P(k+1):\\
			2^(k+1)<((k+1)+1)!
			" < (k+2)!
		\item We have to show that the statement P() holds also for k+1. So we have to solve the inequality of step iii. and show that it is true.
	\end{enumerate}
\end{enumerate}

\end{document}
